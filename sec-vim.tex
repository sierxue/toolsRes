\section{Vim 编辑器}%
\label{sec:vim}

\subsection{参考资料}%
\label{sub:vim-refs}

\begin{itemize}
    \item 官方网址: \href{https://www.vim.org/}{https://www.vim.org/}
    \item 问答网站:
        \href{https://vi.stackexchange.com/}{https://vi.stackexchange.com/}
    \item Vim 教程:
        \begin{itemize}
            \item \lstinline{vimtutor}\footnote{
                在终端中,输入 \lstinline{vimtutor} 命令学习 Vim
                的基本用法。}
            \item \href{https://github.com/wsdjeg/vim-galore-zh_cn}
                {Vim 从入门到精通}
            \item \href{https://item.jd.com/12056490.html}
                {Vim 使用技巧 (Pratical Vim) 第二版}
        \end{itemize}
\end{itemize}

\subsection{常用配置}%
\label{sub:vim-intro}

\subsubsection{Vim 的配置文件}%
\label{ssub:vim-config}

本虚拟机中的配置文件为:
\begin{lstlisting}[style=lst]
~/.df/dotfiles/vimrc
~/.df/dotfiles-local/gvimrc
~/.vimrc_customized
~/.gvimrc_customized
\end{lstlisting}
\lstinline{vimrc} 和 \lstinline{gvimrc} 由
\S\ref{ssub:vbox-guest-ganx-conf} 中的两个 git 仓库控制,请勿修改。
你对本台虚拟机的特殊设置可以在 \lstinline{.vimrc_customized} 和
\lstinline{.gvimrc_customized} 中配置。

\subsubsection{gVim 的个性化设置}%
\label{ssub:vim-gui-config-customized}

打开 \lstinline{~/.gvimrc_customized}, 进行 gVim 的个性化设置。例如:
\begin{lstlisting}[style=lst]
set guifont=Monospace\ 13
set linespace=4
\end{lstlisting}
对于以上两个选项的设置,可以用命令 \lstinline{:h guifont} 和
\lstinline{:h linespace} 查看帮助。
例如,在 \lstinline{guifont} 的设置帮助中,有建议在中文系统下,使用如下配置:
\begin{lstlisting}[xleftmargin=.04\textwidth]
if has("gui_gtk3")
  set guifont=Bitstream\ Vera\ Sans\ Mono\ 12,Fixed\ 12
  set guifontwide=Microsoft\ Yahei\ 12,WenQuanYi\ Zen\ Hei\ 12
endif
\end{lstlisting}
\begin{note}\label{note:guifont}
    在本虚拟机上,这个建议的设置不尽人意。也许在 \lstinline{Window}
    系统上还不错。
\end{note}

\subsection{常用插件}%
\label{sub:vim-plugins}

\subsubsection{vimtex --- 用 Vim 编辑 \LaTeX{} 的利器}%
\label{ssub:vim-plugin-vimtex}

插件网址:
\href{https://github.com/lervag/vimtex}{https://github.com/lervag/vimtex}

常用操作:
\begin{itemize}
    \item \lstinline{\ll} 编译 \LaTeX{} 文档;
    \item \lstinline{\lv} 正向搜索 (Forward search)\footnote{
        打开 pdf 文档并且指向光标所在文本对应的位置。};
    \item \lstinline{\ll} 停止编译 (或者 \lstinline{\lk});
    \item \lstinline{\le} 查看错误和警告信息;
    \item \lstinline{\lc} 清除编译中的辅助文件;
    \item \lstinline{\lC} 清除编译中的辅助文件以及 pdf 文件;
    \item \lstinline{ctrl + clik} 反向搜索 (Backward/inverse search)\footnote{
        按住 ctrl 键,点击 pdf 文档上的内容,光标就会转到 gVim 上相应的位置};
    \item 更多操作见插件帮助文档;
    \item 插件的详细文档可以在命令模式下用 \lstinline{h:vimtex} 命令查看。
\end{itemize}

%待续
%%Vim 的学习曲线(To do)。
%
%\subsection{Vim 的学习方式}%
%\label{sub:vim-learning}

