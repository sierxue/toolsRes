\section{Virtualbox 虚拟机软件}%
\label{sec:vbox}

\subsection{参考资料}%
\label{sub:vbox-refs}

Virtualbox 虚拟机软件参考资料网址:
\begin{itemize}
    \item 官方网址: \href{https://www.virtualbox.org/}{https://www.virtualbox.org/}
    \item 官方论坛: \href{https://forums.virtualbox.org/}{https://forums.virtualbox.org/}
\end{itemize}

\subsection{下载安装}%
\label{sub:vbox-install}
\href{https://www.virtualbox.org/wiki/Downloads}{官方网址}
下载虚拟机软件通常比较慢,可以在
\href{https://mirror.tuna.tsinghua.edu.cn/virtualbox/}{清华镜像}下载。
点击
\href{https://mirror.tuna.tsinghua.edu.cn/virtualbox/LATEST.TXT}
{Virtualbox清华镜像最新版本},可以查到当前最新版本。
写作这个教程的时候,版本是 6.1.
Virtualbox 虚拟机的安装分为两个步骤:
\begin{enumerate}
    \item 在主机(host computer)上安装基本软件;
        \begin{itemize}
            \item \textbf{Windows}: 下载
                \href{https://mirror.tuna.tsinghua.edu.cn/virtualbox/virtualbox-Win-latest.exe}
                {最新 Virtualbox 基本软件},按常规安装。
            \item \textbf{Mac}: 下载
                \href{https://mirror.tuna.tsinghua.edu.cn/virtualbox/virtualbox-osx-latest.dmg}
                {最新 Virtualbox 基本软件},按常规安装。
            \item Linux 的所有发行版可以参考
                \href{https://www.virtualbox.org/wiki/Linux_Downloads}{官方安装文档},
                Debian 系列的发行版可以按下面两种方法安装:
                \begin{itemize}
                    \item 在
                        \href{https://mirror.tuna.tsinghua.edu.cn/virtualbox/}
                        {清华镜像}
                        下载相应的 \lstinline{deb} 安装包安装。
                    \item 参考
                        \href{https://www.virtualbox.org/wiki/Linux_Downloads}
                        {官方安装文档}
                        添加源,用 \lstinline{apt} 方式安装。
                        \begin{note}\label{note:vbox-apt-upgrade}
                           当版本跳跃时,比如从 6.0 升级到 6.1,
                           \lstinline{sudo apt update}
                           并不提醒升级,需要运行命令
                           \lstinline{sudo apt install virtualbox-6.1}.
                        \end{note}
                \end{itemize}
        \end{itemize}
    \item 在主机(host computer)上安装 \lstinline{Extension pack}.
        此步中的 \lstinline{Extension pack} 软件对 Windows, Mac, Linux
        等都通用。
        点击
        \href{https://mirror.tuna.tsinghua.edu.cn/virtualbox/}{清华镜像},
        选择最新的版本号,
        假设最新的版本为 6.1.0,
        则找到并下载
        \begin{center}\label{center:vm-extension}
            \lstinline{Oracle_VM_VirtualBox_Extension_Pack-6.1.0.vbox-extpack},
        \end{center}
        双击后,按提示安装。
\end{enumerate}

\subsection{使用虚拟机}%
\label{sub:vbox-vm}

按照 \S\ref{sub:vbox-install} 完成安装后,
就可以使用虚拟机 (guest machine) 了。
虚拟机的一个优点是共享方便,团队成员使用同样的配置大大有利于团队协作。
为配合本教程,我们精心制作了虚拟机,可以让读者快速地学习常用的开源工具,
而不用花费大量时间于软件的安装和配置。

\newpage
